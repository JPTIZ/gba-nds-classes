\documentclass{beamer}
\usetheme{fibeamer}
\usepackage[main=english,portuguese]{babel}
\selectlanguage{portuguese}

\title{Curso de Desenvolvimento GBA}
\subtitle{1. Introdução ao GBA}
\author{João Paulo Taylor Ienczak Zanette}

\begin{document}

\maketitle

\begin{darkframes}
    \begin{frame}{Índice}
        \tableofcontents
    \end{frame}

    \section{Introdução}
    \subsection{Contextualização do Curso}

    \begin{frame}{Objetivos}
        \begin{itemize}
            \item Ensinar programação de baixo-nível (comunicação direta com
                hardware/integração com assembly);
            \item Ensinar técnicas de programação aplicadas;
            \item Mostrar o funcionamento de imagens/gráficos e áudio no mundo
                digital;
            \item Relacionar as tecnologias vistas com as utilizadas
                atualmente.
        \end{itemize}
    \end{frame}

    \begin{frame}{Programação}
        \begin{itemize}
            \item Assembly ARM7-TDMI --- Modo Thumb (GBA);
            \item OpenGL (NDS);
            \item C++ (GBA/NDS).
        \end{itemize}

        A mesma forma de programação para GBA serve também para: GB, GBC, NES,
        SNES, MegaDrive, SegaSaturn e PSX (PS1).
    \end{frame}

    \begin{frame}{Circuitos e Técnicas Digitais}
        \begin{itemize}
            \item Leitura/escrita de registradores (em que cada bit é mapeado
                para uma função específica) via programação;
        \end{itemize}
    \end{frame}

    \begin{frame}{Sistemas Digitais}
        \begin{itemize}
            \item Compreensão a respeito de como o Assembly gerado pela
                compilação altera o estado/memória do circuito;
            \item Compreensão do sistema que gera imagens em um circuito
                digital (VGA, LCD, etc\ldots);
            \item Funcionamento (inclusive a nível de circuito) da execução de
                músicas em formato de instrução MIDI\@;
            \item Técnicas de otimização através de Hardware.
        \end{itemize}
    \end{frame}

    \begin{frame}{Computação Gráfica}
        \begin{itemize}
            \item Desenho de primitivas (linhas, triângulos, circuitos, etc\ldots);
            \item Aceleração gráfica via Hardware.
        \end{itemize}
    \end{frame}

\end{darkframes}
\end{document}
