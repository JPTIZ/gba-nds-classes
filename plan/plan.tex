\documentclass{article}

\usepackage[utf8]{inputenc}
\usepackage[T1]{fontenc}
\usepackage[portuguese]{babel}

\usepackage[inline]{enumitem}

\usepackage{hyperref}
\usepackage{listing}

\title{Programa do Minicurso}
\author{}
\date{}

\begin{document}

\maketitle

\section{Identificação}

\begin{itemize}
    \item[] Disciplina: Minicurso GameBoyAdvance
    \item[]
        \hspace{-1em}
        \begin{tabular}{ccc}
            Carga Horária: 18 horas-aula&
            Teóricas: 10&
            Práticas: 8
        \end{tabular}
    \item[] Período: Abril a Julho de 2017.
\end{itemize}

\section{Cursos (público alvo)}
\begin{itemize}
    \item[] Ciência da Computação.
\end{itemize}

\section{Requisitos}
\begin{itemize}
    \item[] Mínimo conhecimento a respeito de programação imperativa
        (variáveis, funções, estruturas condicionais/de repetição,\ldots) e
        circuitos (o que é um registrador, sinais de entrada/saída,\ldots).
\end{itemize}

\section{Ementa}
\begin{itemize*}[label={}]
    % GBA-specific
    \item Especificações técnicas e limitações do GBA (\textit{GameBoyAdvance}); % DONE
    \item Arquitetura do processador ARM7TDMI\@; % DONE
    \item ISA (\textit{Instruction Set Architecture}) ARMv4; % DONE
    \item Assembly ARM e Thumb; % DONE
    % Game console-specific
    \item VRAM\@ (\textit{Video-RAM}); % DONE
    \item OAM (\textit{Object-Attribute-Memory}); % DONE
    % Circuitos / Sistemas Digitais
    \item Registradores de controle; % DONE - Alguns
    \item \textit{Input/Keypad}; % DONE
    \item \textit{Hardware-Rendering}; % DONE
    % Org
    \item \textit{BIOS-Call}; % DONE
    \item \textit{Hardware-Interrupts}; % DONE
    \item DMA (\textit{Dynamic-Memory-Access}); % DONE
    \item \textit{Endiannesss}; % DONE
    % SO
    \item Gerenciamento de memória para Sistemas Embarcados; % DONE
    \item Otimizações para Sistemas Embarcados; % DONE
    % Gráficos
    \item \textit{Spritesetting}; % DONE
    \item \textit{Display-Modes} de consoles; % DONE
    \item Matriz de transformação; % DONE
    \item \textit{V-Sync}; % DONE
    \item \textit{Windowing}; % DONE
    \item Processamento de imagem; % DONE
    % Técnicas de otimização
    \item \textit{Palette-Swapping}; % DONE
    \item \textit{Page-Flipping}; % DONE
    \item \textit{Tile-Mapping}; % DONE
    \item Economia de energia; % DONE
    % Geral
    \item Fontes; % DONE
    \item MIDI\@; % DONE
    \item Ondas sonoras; % DONE
    \item Processamento de arquivos de mídia (imagem/áudio); % DONE
\end{itemize*}

\section{Objetivos}

\begin{description}
    \item[Geral:] Entusiasmar alunos de graduação a respeito de sistemas
        embarcados, mídia e otimizações de baixo nível, mostrando uma aplicação
        diferente do usual e apresentando o funcionamento e interação do
        \textit{Hardware} envolvido.
    \item[Específico:] São objetivos do minicurso:
        \begin{itemize}[label={-}]
            \item Apresentar as especificações técnicas da plataforma;
            \item Mencionar as limitações da plataforma dentro do seu contexto
                histórico;
            \item Apresentar a interação entre os registradores de controle e
                suas respectivas funcionalidades a nível de \textit{Hardware};
            \item Apresentar o funcionamento de mídia analógica/digital;
            \item Apresentar otimizações válidas para sistemas embarcados, bem
                como tecnologias envolvidas nos mais modernos;
            \item Prover pleno entendimento da integração entre programação de
                alto-nível e o \textit{Hardware};
            \item Apresentar paralelos entre a tecnologia ensinada e as
                atualmente utilizadas;
            \item Apresentar o uso de C++ moderno em sistemas embarcados, bem
                como suas vantagens/desvantagens.
        \end{itemize}
\end{description}

\section{Conteúdo Programático}

\begin{enumerate}
    \item Introdução ao GBA (2 horas):
        \begin{enumerate}
            \item Histórico de limitação dos consoles pré-2000;
            \item Especificações técnicas do GBA\@;
            \item Mapeamento de memória do GBA\@;
            \item VRAM\@;
            \item Assembly Thumb;
            \item \textit{Mode 3};
            \item Introdução a C++.
        \end{enumerate}
    \item \textit{Bitmapped Display-Modes} (2 horas):
        \begin{enumerate}
            \item Visão geral de \textit{Bitmapped Modes};
            \item PPU (\textit{Pixel Processing Unit});
            \item Análise da Performance do \textit{Mode 3};
            \item \textit{V-Sync};
            \item \textit{Mode 5};
            \item \textit{Page-Flipping};
            \item \textit{Mode 4};
            \item \textit{Paletting};
            \item \textit{Endianness}.
        \end{enumerate}
    \item Objetos de jogo e \textit{Input} (2 horas):
        \begin{enumerate}
            \item Leitura do \textit{Keypad};
            \item \textit{Sprites};
            \item OAM (\textit{Object Attribute Memory});
            \item Múltiplos objetos;
            \item Introdução a arquivos de \textit{Bitmap}.
        \end{enumerate}
    \item Arquivos \textit{Bitmap} (2 horas):
        \begin{enumerate}
            \item \textit{std::ifstream} e \textit{std::ofstream};
            \item Header de tipo de arquivo;
            \item Header BMP\@;
        \end{enumerate}
    \item \textit{Character Display-Modes} (2 horas):
        \begin{enumerate}
            \item \textit{Tilemapping};
            \item \textit{Palette-Swapping};
        \end{enumerate}
    \item Funcionalidades e Efeitos gráficos (2 horas):
        \begin{enumerate}
            \item \textit{Windowing};
            \item \textit{Alpha-Blend};
            \item Luminosidade;
            \item Efeito mosáico;
            \item Matriz de transformação.
        \end{enumerate}
    \item Texto, Economia de energia e Cartucho (2 horas):
        \begin{enumerate}
            \item Fontes;
            \item Janelas de diálogo;
            \item \textit{Hardware-Interrupts};
            \item DMA --- \textit{Dynamic Memory Access};
            \item \textit{SaveRAM};
        \end{enumerate}
    \item Áudio (2 horas):
        \begin{enumerate}
            \item Arquivos MIDI\@;
            \item Ondas sonoras;
            \item Interferência construtiva/destrutiva;
            \item \textit{BIOS-Calls} para áudio;
            \item Efeitos sonoros (SFX) e arquivos WAV\@.
        \end{enumerate}
    \item Construção de um jogo (2 horas):
        \begin{enumerate}
            \item \textit{Splash Screen};
            \item Eventos e animações;
            \item Gerência de memória para sistemas embarcados;
            \item Boas práticas de desenvolvimento.
        \end{enumerate}
\end{enumerate}

\section{Cronograma}

\begin{enumerate}[label= (\alph*)]
    \item Primeiro encontro:
        \begin{enumerate}
            \item Introdução ao GBA\@;
            \item Apresentação da Toolchain DevKitPro;
            \item Apresentação das ferramentas de desenvolvedor do emulador
                \textit{VisualBoyAdvance}.
        \end{enumerate}
    \item Segundo encontro:
        \begin{enumerate}
            \item \textit{Bitmapped Display-Modes};
            \item Criação de uma pequena biblioteca, momentameamente lidando
                apenas com acesso a registradores/memória do GBA\@.
        \end{enumerate}
    \item Terceiro encontro:
        \begin{enumerate}
            \item Objetos de jogo e \textit{Input};
            \item Adicionar funções à biblioteca principal para lidar com
                cores, objetos de jogo e sprites.
            \item Adicionar funções à biblioteca principal para lidar com
                \textit{Input} de jogo.
        \end{enumerate}
    \item Quarto encontro:
        \begin{enumerate}
            \item Arquivos \textit{Bitmap};
            \item Criação de um conversor de arquivos \textit{Bitmap} para
                formatos usáveis no GBA\@.
        \end{enumerate}
    \item Quinto encontro:
        \begin{enumerate}
            \item \textit{Character Display-Modes};
            \item Adicionar módulo para lidar com diferentes telas de jogo à
                biblioteca principal.
        \end{enumerate}
    \item Sexto encontro:
        \begin{enumerate}
            \item Funcionalidades e Efeitos gráficos;
            \item Adicionar módulo de efeitos gráficos à biblioteca principal;
            \item Adicionar módulo de \textit{Windowing} à biblioteca principal.
        \end{enumerate}
    \item Sétimo encontro:
        \begin{enumerate}
            \item Texto, Economia de energia e Cartucho;
            \item Adicionar módulo de fontes à biblioteca principal;
            \item Adicionar módulo para DMA à biblioteca principal.
        \end{enumerate}
    \item Oitavo encontro:
        \begin{enumerate}
            \item Áudio;
            \item Adicionar módulo de áudio (músicas e SFX) à biblioteca
                principal.
        \end{enumerate}
    \item Nono encontro:
        \begin{enumerate}
            \item Construção de um jogo;
            \item Considerações a respeito de uso de globais, APIs, dentre
                outros recursos de programação.
        \end{enumerate}
\end{enumerate}

\nocite{*}
\bibliographystyle{unsrt}
\bibliography{plan}

\end{document}
