\documentclass{article}

\usepackage[utf8]{inputenc}
\usepackage[T1]{fontenc}
\usepackage[portuguese]{babel}

\usepackage[inline]{enumitem}

\usepackage{hyperref}
\usepackage{listing}

\title{Programa do Minicurso}
\author{}
\date{}

\begin{document}

\maketitle

\section{Identificação}

\begin{itemize}
    \item[] Disciplina: Minicurso GameBoyAdvance
    \item[]
        \hspace{-1em}
        \begin{tabular}{ccc}
            Carga Horária: 20 horas-aula&
            Teóricas: 10&
            Práticas: 12
        \end{tabular}
    \item[] Período: Março a Maio de 2017.
\end{itemize}

\section{Cursos (público alvo)}
\begin{itemize}
    \item[] Ciência da Computação.
\end{itemize}

\section{Requisitos}
\begin{itemize}
    \item[] Mínimo conhecimento a respeito de programação imperativa
        (variáveis, funções, estruturas condicionais/de repetição,\ldots) e
        circuitos (o que é um registrador, sinais de entrada/saída,\ldots).
\end{itemize}

\section{Ementa}
\begin{itemize*}[label={}]
    % GBA-specific
    \item Especificações técnicas e limitações do GBA (\textit{GameBoyAdvance});
    \item Arquitetura do processador ARM7TDMI\@;
    \item ISA (\textit{Instruction Set Architecture}) ARMv4;
    \item Assembly ARM-Thumb;
    % Game console-specific
    \item VRAM\@;
    \item OAM (\textit{Object-Attribute-Memory});
    % Circuitos / Sistemas Digitais
    \item Registradores de controle;
    \item \textit{Hardware-Rendering};
    % Org
    \item \textit{BIOS-Call};
    \item \textit{Hardware-Interrupts};
    \item DMA (\textit{Dynamic-Memory-Access});
    % SO
    \item Gerenciamento de memória manual;
    \item Otimizações para sistemas embarcados;
    % Técnicas de otimização
    \item \textit{Palette-Swapping};
    \item \textit{Page-Flipping};
    \item \textit{Tile-Mapping};
    \item \textit{Spritesetting};
    % Geral
    \item \textit{Display-Modes} de consoles;
    \item MIDI\@;
    \item Ondas sonoras;
    \item Processamento de arquivos de mídia;
    \item Economia de energia;
    \item Matriz de transformação;
\end{itemize*}

\section{Objetivos}

\begin{description}
    \item[Geral:] Entusiasmar alunos de graduação a respeito de sistemas
        embarcados, mídia e otimizações de baixo nível, mostrando uma aplicação
        diferente do usual e apresentando o funcionamento e interação do
        \textit{Hardware} envolvido.
    \item[Específico:] São objetivos do minicurso:
        \begin{itemize}[label={-}]
            \item Apresentar as especificações técnicas da plataforma;
            \item Mencionar as limitações da plataforma dentro do seu contexto
                histórico;
            \item Apresentar a interação entre os registradores de controle e
                suas respectivas funcionalidades a nível de \textit{Hardware};
            \item Apresentar o funcionamento de mídia analógica/digital;
            \item Apresentar otimizações válidas para sistemas embarcados, bem
                como tecnologias envolvidas nos mais modernos;
            \item Prover pleno entendimento da integração entre programação de
                alto-nível e o \textit{Hardware};
            \item Apresentar paralelos entre a tecnologia ensinada e as
                atualmente utilizadas;
            \item Apresentar o uso de C++ moderno em sistemas embarcados, bem
                como suas vantagens/desvantagens.
        \end{itemize}
\end{description}

\section{Conteúdo Programático}

\begin{enumerate}
    \item Introdução ao GBA (2 horas):
        \begin{enumerate}
            \item Histórico de limitação dos consoles pré-2000;
            \item Especificações técnicas do GBA\@;
            \item Mapeamento de memória do GBA\@;
            \item VRAM\@;
            \item Assembly Thumb;
            \item \textit{Mode 3};
            \item Introdução a C++.
        \end{enumerate}
    \item \textit{Bitmap Display-Modes} (2 horas):
        \begin{enumerate}
            \item Visão geral de \textit{Bitmap Modes};
            \item PPU (\textit{Pixel Processing Unit});
            \item Performance do \textit{Mode 3};
            \item \textit{V-Sync};
            \item \textit{Mode 4};
            \item \textit{Paletting}.
            \item \textit{Page-Flipping};
        \end{enumerate}
    \item Objetos de jogo e Input (2 horas):
        \begin{enumerate}
            \item Leitura do \textit{Keypad};
            \item \textit{Sprites};
            \item OAM (\textit{Object Attribute Memory});
        \end{enumerate}
\end{enumerate}

\section{Cronograma}

\begin{enumerate}[label= (\alph*)]
    \item Primeiro encontro:
        \begin{enumerate}
            \item Introdução ao GBA\@;
            \item Apresentação da Toolchain DevKitPro;
            \item Apresentação das ferramentas de desenvolvedor do emulador
                \textit{VisualBoyAdvance}.
        \end{enumerate}
\end{enumerate}

\nocite{*}
\bibliographystyle{unsrt}
\bibliography{plan}

\end{document}
