\documentclass{article}

\usepackage[utf8]{inputenc}
\usepackage[T1]{fontenc}
\usepackage[portuguese]{babel}

\usepackage[inline]{enumitem}

\usepackage{hyperref}
\usepackage{listing}

\title{Programa do Minicurso}
\author{}
\date{}

\begin{document}

\maketitle

\section{Identificação}

\begin{itemize}
    \item[] Disciplina: Minicurso De Desenvolvimento Para Game Boy Advance
    \item[]
        \hspace{-1em}
        \begin{tabular}{ccc}
            Carga Horária: 18 horas-aula&
            Teóricas: 10&
            Práticas: 8
        \end{tabular}
    \item[] Período: Setembro a Novembro de
                     2017.
\end{itemize}

\section{Cursos (público alvo)}
\begin{itemize}
    \item[] Ciência da Computação;
    \item[] Interessados em computação embarcada.
\end{itemize}

\section{Requisitos}
\begin{itemize}
    \item[] Mínimo conhecimento a respeito de programação imperativa (variáveis, funções, estruturas condicionais/de repetição, ...) e circuitos (o que é um registrador, sinais de entrada/saída, ...).
\end{itemize}

\section{Ementa}
\begin{itemize*}[label={}]
    \item[] Histórico de limitação dos consoles pré-2000;
    \item[] Especificações técnicas do GBA;
    \item[] Mapeamento de memória do GBA;
    \item[] VRAM;
    \item[] Assembly Thumb;
    \item[] \textit{Mode 3};
    \item[] Introdução a C++.
    \item[] Visão geral de Modos Bitmapeados;
    \item[] PPU (\textit{Pixel Processing Unit});
    \item[] Análise da Performance do \textit{Mode 3};
    \item[] \textit{V-Sync};
    \item[] \textit{Mode 5};
    \item[] \textit{Page-Flipping};
    \item[] \textit{Mode 4};
    \item[] \textit{Paletting};
    \item[] \textit{Endianness}.
    \item[] Leitura do \textit{Keypad};
    \item[] \textit{Sprites};
    \item[] OAM (\textit{Object Attribute Memory});
    \item[] Múltiplos objetos;
    \item[] Introdução a arquivos de \textit{Bitmap}.
    \item[] \textit{std::ifstream} e \textit{std::ofstream};
    \item[] Header de tipo de arquivo;
    \item[] Header BMP.
    \item[] \textit{Tilemapping};
    \item[] \textit{Palette-Swapping}.
    \item[] \textit{Windowing};
    \item[] \textit{Alpha-Blend};
    \item[] Luminosidade;
    \item[] Efeito mosáico;
    \item[] Matriz de transformação.
    \item[] Fontes;
    \item[] Janelas de diálogo;
    \item[] \textit{Hardware-Interrupts};
    \item[] DMA - \textit{Dynamic Memory Access};
    \item[] \textit{SaveRAM}.
    \item[] Arquivos MIDI;
    \item[] Ondas sonoras;
    \item[] Interferência construtiva/destrutiva;
    \item[] \textit{BIOS-Calls} para áudio;
    \item[] Efeitos sonoros (SFX) e arquivos WAV.
    \item[] \textit{Splash Screen}.
    \item[] Eventos e animações.
    \item[] Gerência de memória para sistemas embarcados.
    \item[] Boas práticas de desenvolvimento.
\end{itemize*}

\section{Objetivos}

\begin{description}
    \item[Geral:] Entusiasmar alunos de graduação a respeito de sistemas embarcados, mídia e otimizações de baixo nível, mostrando uma aplicação diferente do habitual e apresentando o funcionamento e interação do Hardware \textit{Hardware} envolvido.
    \item[Específico:] São objetivos do minicurso:
        \begin{itemize}[label={-}]
                \item[] Apresentar as especificações técnicas da plataforma;
                \item[] Mencionar as limitações da plataforma dentro do seu  contexto histórico;
                \item[] Apresentar a interação entre os registradores de controle e suas respectivas funcionalidades a nível de \textit{Hardware};
                \item[] Apresentar o funcionamento de mídia analógica/digital;
                \item[] Apresentar otimizações válidas para sistemas embarcados, bem como tecnologias envolvidas nos mais modernos;
                \item[] Prover pleno entendimento da integração entre programação de alto-nível e o \textit{Hardware};
                \item[] Apresentar paralelos entre a tecnologia ensinada e as atualmente utilizadas;
                \item[] Apresentar o uso de C++ moderno em sistemas embarcados, bem como suas vantagens e desvantagens.
        \end{itemize}
\end{description}

\section{Conteúdo Programático}

\begin{enumerate}
    \item Introdução ao GBA (2 horas):
        \begin{enumerate}
                \item Histórico de limitação dos consoles pré-2000;
                \item Especificações técnicas do GBA;
                \item Mapeamento de memória do GBA;
                \item VRAM;
                \item Assembly Thumb;
                \item \textit{Mode 3};
                \item Introdução a C++.
        \end{enumerate}
    \item Modos de Visor Bitmapeados (2 horas):
        \begin{enumerate}
                \item Visão geral de Modos Bitmapeados;
                \item PPU (\textit{Pixel Processing Unit});
                \item Análise da Performance do \textit{Mode 3};
                \item \textit{V-Sync};
                \item \textit{Mode 5};
                \item \textit{Page-Flipping};
                \item \textit{Mode 4};
                \item \textit{Paletting};
                \item \textit{Endianness}.
        \end{enumerate}
    \item Objetos de jogo e \textit{Input} (2 horas):
        \begin{enumerate}
                \item Leitura do \textit{Keypad};
                \item \textit{Sprites};
                \item OAM (\textit{Object Attribute Memory});
                \item Múltiplos objetos;
                \item Introdução a arquivos de \textit{Bitmap}.
        \end{enumerate}
    \item Arquivos \textit{Bitmap} (2 horas):
        \begin{enumerate}
                \item \textit{std::ifstream} e \textit{std::ofstream};
                \item Header de tipo de arquivo;
                \item Header BMP.
        \end{enumerate}
    \item Modos de Visor em Caractere (2 horas):
        \begin{enumerate}
                \item \textit{Tilemapping};
                \item \textit{Palette-Swapping}.
        \end{enumerate}
    \item Funcionalidades e Efeitos Gráficos (2 horas):
        \begin{enumerate}
                \item \textit{Windowing};
                \item \textit{Alpha-Blend};
                \item Luminosidade;
                \item Efeito mosáico;
                \item Matriz de transformação.
        \end{enumerate}
    \item Texto, Economia de energia e Cartucho (2 horas):
        \begin{enumerate}
                \item Fontes;
                \item Janelas de diálogo;
                \item \textit{Hardware-Interrupts};
                \item DMA - \textit{Dynamic Memory Access};
                \item \textit{SaveRAM}.
        \end{enumerate}
    \item Áudio (2 horas):
        \begin{enumerate}
                \item Arquivos MIDI;
                \item Ondas sonoras;
                \item Interferência construtiva/destrutiva;
                \item \textit{BIOS-Calls} para áudio;
                \item Efeitos sonoros (SFX) e arquivos WAV.
        \end{enumerate}
    \item Construção de um jogo (2 horas):
        \begin{enumerate}
                \item \textit{Splash Screen};
                \item Eventos e animações;
                \item Gerência de memória para sistemas embarcados;
                \item Boas práticas de desenvolvimento.
        \end{enumerate}
\end{enumerate}

\section{Cronograma}

O cronograma segue a mesma ordenação do Conteúdo Programático, sendo cada item
uma aula separada.

\nocite{*}
\bibliographystyle{unsrt}
\bibliography{plan}

\end{document}