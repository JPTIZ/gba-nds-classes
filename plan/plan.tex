\documentclass{article}

\usepackage[utf8]{inputenc}
\usepackage[T1]{fontenc}
\usepackage[portuguese]{babel}

\usepackage[inline]{enumitem}

\usepackage{hyperref}
\usepackage{listing}

\title{Programa do Minicurso}
\author{}
\date{}

\begin{document}

\maketitle

\section{Identificação}

\begin{itemize}
    \item[] Disciplina: Minicurso De Desenvolvimento Para Game Boy Advance
    \item[]
        \hspace{-1em}
        \begin{tabular}{ccc}
            Carga Horária: 18 horas-aula&
            Teóricas: 10&
            Práticas: 8
        \end{tabular}
    \item[] Período: Setembro a Novembro de
                     2017.
\end{itemize}

\section{Cursos (público alvo)}
\begin{itemize}
    \item[] Ciência da Computação;
    \item[] Interessados em computação embarcada.
\end{itemize}

\section{Requisitos}
\begin{itemize}
    \item[] Mínimo conhecimento a respeito de programação imperativa (variáveis, funções, estruturas condicionais/de repetição, ...) e circuitos (o que é um registrador, sinais de entrada/saída, ...).
\end{itemize}

\section{Ementa}
\begin{itemize*}[label={}]
    \item[] Histórico de limitação dos consoles pré-2000;
    \item[] Especificações técnicas do GBA;
    \item[] Mapeamento de memória do GBA;
    \item[] VRAM;
    \item[] Assembly Thumb;
    \item[] \textit{Mode 3};
    \item[] Introdução a C++.
    \item[] PPU (\textit{Pixel Processing Unit});
    \item[] \textit{V/H-Blank} e \textit{V-Sync};
    \item[] \textit{Modes 4/5};
    \item[] \textit{Page-Flipping};
    \item[] \textit{Paletting};
    \item[] \textit{Endianness};
    \item[] Performance de modos Bitmapeados.
    \item[] \textit{Mode 0};
    \item[] \textit{Tilemapping};
    \item[] Técnica de \textit{Palette-Swapping};
    \item[] Múltiplas camadas de mapa;
    \item[] Deslocamento de camada.
    \item[] \textit{Windowing};
    \item[] \textit{Alpha-Blend};
    \item[] Luminosidade;
    \item[] Efeito mosáico.
    \item[] Leitura do \textit{Keypad};
    \item[] \textit{Sprites};
    \item[] OAM (\textit{Object Attribute Memory});
    \item[] Múltiplos objetos.
    \item[] Header de arquivo;
    \item[] Header BMP e formato de imagem \textit{Bitmap};
    \item[] Introdução a Python 3;
    \item[] Bibliotecas ferramental \textit{carl};
    \item[] Biblioteca de imagens \textit{PIL}.
    \item[] Interrupções de \textit{Hardware/Software};
    \item[] Tratadores de interrupções;
    \item[] Aspectos avançados da \textit{BIOS}.
    \item[] Sistema MIDI e Efeitos sonoros (SFX);
    \item[] Ondas sonoras e interferência construtiva/destrutiva;
    \item[] Introdução à Teoria Musical;
    \item[] Canais de áudio e limitações;
    \item[] \textit{Direct Memory Access} (DMA).
    \item[] Fonte e texto.
    \item[] Organização dos estados de jogo.
    \item[] \textit{SaveRAM}.
    \item[] Gerência de memória para sistemas embarcados.
\end{itemize*}

\section{Objetivos}

\begin{description}
    \item[Geral:] Entusiasmar alunos de graduação a respeito de sistemas embarcados, mídia e otimizações de baixo nível, mostrando uma aplicação diferente do habitual e apresentando o funcionamento e interação do \textit{Hardware} envolvido.
    \item[Específico:] São objetivos do minicurso:
        \begin{itemize}[label={-}]
                \item Apresentar as especificações técnicas da plataforma;
                \item Mencionar as limitações da plataforma dentro do seu  contexto histórico;
                \item Apresentar a interação entre os registradores de controle e suas respectivas funcionalidades a nível de \textit{Hardware};
                \item Apresentar o funcionamento de mídia analógica/digital;
                \item Apresentar otimizações válidas para sistemas embarcados, bem como tecnologias envolvidas nos mais modernos;
                \item Prover pleno entendimento da integração entre programação de alto-nível e o \textit{Hardware};
                \item Apresentar paralelos entre a tecnologia ensinada e as atualmente utilizadas;
                \item Apresentar o uso de C++ moderno em sistemas embarcados, bem como suas vantagens e desvantagens.
        \end{itemize}
\end{description}

\section{Conteúdo Programático}

\begin{enumerate}
    \item Introdução ao GBA (2 horas):
        \begin{samepage}
        \begin{enumerate}
                \item Histórico de limitação dos consoles pré-2000;
                \item Especificações técnicas do GBA;
                \item Mapeamento de memória do GBA;
                \item VRAM;
                \item Assembly Thumb;
                \item \textit{Mode 3};
                \item Introdução a C++.
        \end{enumerate}
        \end{samepage}
    \item Modos de Visor Bitmapeados (2 horas):
        \begin{samepage}
        \begin{enumerate}
                \item PPU (\textit{Pixel Processing Unit});
                \item \textit{V/H-Blank} e \textit{V-Sync};
                \item \textit{Modes 4/5};
                \item \textit{Page-Flipping};
                \item \textit{Paletting};
                \item \textit{Endianness};
                \item Performance de modos Bitmapeados.
        \end{enumerate}
        \end{samepage}
    \item Modos de Visor em Caractere (2 horas):
        \begin{samepage}
        \begin{enumerate}
                \item \textit{Mode 0};
                \item \textit{Tilemapping};
                \item Técnica de \textit{Palette-Swapping};
                \item Múltiplas camadas de mapa;
                \item Deslocamento de camada.
        \end{enumerate}
        \end{samepage}
    \item Efeitos gráficos (2 horas):
        \begin{samepage}
        \begin{enumerate}
                \item \textit{Windowing};
                \item \textit{Alpha-Blend};
                \item Luminosidade;
                \item Efeito mosáico.
        \end{enumerate}
        \end{samepage}
    \item Objetos de jogo e \textit{Input} (2 horas):
        \begin{samepage}
        \begin{enumerate}
                \item Leitura do \textit{Keypad};
                \item \textit{Sprites};
                \item OAM (\textit{Object Attribute Memory});
                \item Múltiplos objetos.
        \end{enumerate}
        \end{samepage}
    \item Ferramental: conversor de imagem (2 horas):
        \begin{samepage}
        \begin{enumerate}
                \item Header de arquivo;
                \item Header BMP e formato de imagem \textit{Bitmap};
                \item Introdução a Python 3;
                \item Bibliotecas ferramental \textit{carl};
                \item Biblioteca de imagens \textit{PIL}.
        \end{enumerate}
        \end{samepage}
    \item Economia de energia (2 horas):
        \begin{samepage}
        \begin{enumerate}
                \item Interrupções de \textit{Hardware/Software};
                \item Tratadores de interrupções;
                \item Aspectos avançados da \textit{BIOS}.
        \end{enumerate}
        \end{samepage}
    \item Áudio e transferência de dados (2 horas):
        \begin{samepage}
        \begin{enumerate}
                \item Sistema MIDI e Efeitos sonoros (SFX);
                \item Ondas sonoras e interferência construtiva/destrutiva;
                \item Introdução à Teoria Musical;
                \item Canais de áudio e limitações;
                \item \textit{Direct Memory Access} (DMA).
        \end{enumerate}
        \end{samepage}
    \item Outros aspectos de \textit{Hardware} e Desenvolvimento de jogos (2 horas):
        \begin{samepage}
        \begin{enumerate}
                \item Fonte e texto;
                \item Organização dos estados de jogo;
                \item \textit{SaveRAM};
                \item Gerência de memória para sistemas embarcados.
        \end{enumerate}
        \end{samepage}
\end{enumerate}

\section{Cronograma}

O cronograma segue a mesma ordenação do Conteúdo Programático, sendo cada item
uma aula separada.

\nocite{*}
\bibliographystyle{unsrt}
\bibliography{plan}

\end{document}