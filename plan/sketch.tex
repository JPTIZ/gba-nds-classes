\documentclass{article}

\usepackage[utf8]{inputenc}
\usepackage[T1]{fontenc}
\usepackage[portuguese]{babel}

\usepackage[inline]{enumitem}

\usepackage{hyperref}
\usepackage{listing}

\title{Programa do \VAR{course.kind.title()}}
\author{}
\date{}

\begin{document}

\maketitle

\section{Identificação}

\begin{itemize}
    \item[] Disciplina: \VAR{course.title.title()}
    \item[]
        \hspace{-1em}
        \begin{tabular}{ccc}
            Carga Horária: \VAR{course.hours['theoretical'] +
                                course.hours['practice']} horas-aula&
            Teóricas: \VAR{course.hours['theoretical']}&
            Práticas: \VAR{course.hours['practice']}
        \end{tabular}
    \item[] Período: \VAR{lang[course.start]} a \VAR{lang[course.end]} de
                     \VAR{course.year}.
\end{itemize}

\section{Cursos (público alvo)}
\begin{itemize}
%% for target, more in course.targets|lookahead
%%   if more
    \item[] \VAR{target};
%%   else
    \item[] \VAR{target}.
%%   endif
%% endfor
\end{itemize}

\section{Requisitos}
\begin{itemize}
%% for require, more in course.requires|lookahead
%%   if more
    \item[] \VAR{require};
%%   else
    \item[] \VAR{require}.
%%   endif
%% endfor
\end{itemize}

\section{Ementa}
\begin{itemize*}[label={}]
%% for topic, more_topics in course.topics|lookahead
%%   for item, more_items in topic.items
%%     if more_topics and more_items
    \item[] \VAR{item};
%%     else
    \item[] \VAR{item}.
%%     endif
%%   endfor
%% endfor
\end{itemize*}

\section{Objetivos}

\begin{description}
    \item[Geral:] Entusiasmar alunos de graduação a respeito de sistemas
        embarcados, mídia e otimizações de baixo nível, mostrando uma aplicação
        diferente do usual e apresentando o funcionamento e interação do
        \textit{Hardware} envolvido.
    \item[Específico:] São objetivos do minicurso:
        \begin{itemize}[label={-}]
            \item Apresentar as especificações técnicas da plataforma;
            \item Mencionar as limitações da plataforma dentro do seu contexto
                histórico;
            \item Apresentar a interação entre os registradores de controle e
                suas respectivas funcionalidades a nível de \textit{Hardware};
            \item Apresentar o funcionamento de mídia analógica/digital;
            \item Apresentar otimizações válidas para sistemas embarcados, bem
                como tecnologias envolvidas nos mais modernos;
            \item Prover pleno entendimento da integração entre programação de
                alto-nível e o \textit{Hardware};
            \item Apresentar paralelos entre a tecnologia ensinada e as
                atualmente utilizadas;
            \item Apresentar o uso de C++ moderno em sistemas embarcados, bem
                como suas vantagens/desvantagens.
        \end{itemize}
\end{description}

\section{Conteúdo Programático}

\begin{enumerate}
    \item Introdução ao GBA (2 horas):
        \begin{enumerate}
            \item Histórico de limitação dos consoles pré-2000;
            \item Especificações técnicas do GBA\@;
            \item Mapeamento de memória do GBA\@;
            \item VRAM\@;
            \item Assembly Thumb;
            \item \textit{Mode 3};
            \item Introdução a C++.
        \end{enumerate}
    \item \textit{Bitmapped Display-Modes} (2 horas):
\end{enumerate}

\section{Cronograma}

\begin{enumerate}[label= (\alph*)]
    \item Primeiro encontro:
        \begin{enumerate}
            \item Introdução ao GBA\@;
            \item Apresentação da Toolchain DevKitPro;
            \item Apresentação das ferramentas de desenvolvedor do emulador
                \textit{VisualBoyAdvance}.
        \end{enumerate}
\end{enumerate}

\nocite{*}
\bibliographystyle{unsrt}
\bibliography{plan}

\end{document}

